% Copyright (C) 2024 Elkeid-me
%
% This file is part of Xenon ATC-X.
%
% Xenon ATC-X is free software: you can redistribute it and/or modify
% it under the terms of the GNU General Public License as published by
% the Free Software Foundation, either version 3 of the License, or
% (at your option) any later version.
%
% Xenon ATC-X is distributed in the hope that it will be useful,
% but WITHOUT ANY WARRANTY; without even the implied warranty of
% MERCHANTABILITY or FITNESS FOR A PARTICULAR PURPOSE.  See the
% GNU General Public License for more details.
%
% You should have received a copy of the GNU General Public License
% along with Xenon ATC-X.  If not, see <http://www.gnu.org/licenses/>.

\section{概述}
Xenon 是某大学编译原理课程的作业,其开发目标是把 SysY 编译为 RISC-V 汇编。SysY 是 C 语言的一个子集。

Xenon 已被废弃,Xenon ATC-X(本项目)是其后继项目。ATC-X = \textbf{A}dvanced \textbf{T}echnology \textbf{C}omplier-\textbf{X}(先进技术编译器-X)。
\subsection{基本功能}
将\textbf{扩展的} SysY 编译到 Koopa IR,或 RISC-V(RV32IM)汇编。
\subsection{主要特点}
\begin{enumerate}
    \item 在前端使用\textbf{解析表达文法}而不是\textbf{上下文无关文法},不用处理悬垂 \inlinec{else} 问题。
    \item 语言扩展,将在 \ref{lang-ext} 详细介绍。
\end{enumerate}
\section{编译器设计}
\subsection{主要模块构成}
编译器包含三个主要模块:
\begin{enumerate}
    \item 预处理器,负责换行符替换(见 \ref{stage-1})、行拼接(见 \ref{stage-2})和注释去除。
    \item 前端,负责将不包含注释的 SysY 源代码翻译为 Koopa IR。前端又可以细分为解析器和 IR 生成器。
    \item 后端,负责将 Koopa IR 翻译为 RISC-V 汇编(RV32IM)。
\end{enumerate}
\subsection{主要数据结构}
\subsubsection{抽象语法树}

语法树的根是 \inlinec{TranslationUnit}(翻译单元),\inlinec{TranslationUnit} 实质上是由 \inlinec{Definition}(符号定义)构成的列表。\inlinec{Definition} 是一个元组,包含符号的\textit{标识符}、\textit{初始化器}和符号的\textit{类型}。

\textit{初始化器}是一个可区分联合,分为:
\begin{itemize}
    \item \inlinerust{Function(Vec<Option<String>>, Block)}(函数),其中 \inlinerust{Vec<Option<String>>} 记录了各个参数的标识符(可选),\inlinerust{Block} 存储了函数体;
    \item \inlinerust{Expr(Expr)}(表达式);
    \item \inlinerust{Const(i32)}(常量);
    \item \inlinerust{List(InitList)}(初始化列表);
    \item \inlinerust{ConstList(ConstInitList)}(常量初始化列表)。
\end{itemize}

\textit{类型}也是一个可区分联合,分为
\begin{itemize}
    \item \inlinerust{Int}(整型);
    \item \inlinerust{Void}(空);
    \item \inlinerust{IntPointer(Vec<usize>)}(指针),\inlinerust{Vec<usize>} 存储了各个维度的长度;
    \item \inlinerust{IntArray(Vec<usize>)}(数组),\inlinerust{Vec<usize>} 存储了各个维度的长度;
    \item \inlinerust{Function(Box<Type>, Vec<Type>)}(函数),\inlinerust{Box<Type>} 存储返回值类型,而 \inlinerust{Vec<Type>} 存储各个参数的类型;
\end{itemize}

初始化器和类型\textbf{共同决定}了一个符号。例如:
\begin{itemize}
    \item \inlinec{int a;} 定义的符号 \inlinec{a}
    \begin{itemize}
        \item 初始化器为 \inlinerust{None}
        \item 类型为 \inlinerust{Int}
    \end{itemize}
    \item \inlinec{int a = 1;} 定义的符号 \inlinec{a}
    \begin{itemize}
        \item 初始化器为 \inlinerust{Expr(1)}
        \item 类型为 \inlinerust{Int}
    \end{itemize}
    \item \inlinec{const int a = 1;} 定义的符号 \inlinec{a}
    \begin{itemize}
        \item 初始化器为 \inlinerust{Const(1)}
        \item 类型为 \inlinerust{Int}
    \end{itemize}
    \item \inlinec{int a[2][1] = {{1}, {2}};} 定义的符号 \inlinec{a}
    \begin{itemize}
        \item 初始化器为 \\\inlinerust{List(InitList([InitList([Expr(1)]), InitList([Expr(2)])]))}
        \item 类型为 \inlinerust{IntArray([2][1])}
    \end{itemize}
    \item \inlinec{const int a[2][1] = {{1}, {2}};} 定义的符号 \inlinec{a}
    \begin{itemize}
        \item 初始化器为 \\
        \inlinerust{ConstList(ConstInitList([ConstInitList([1]), ConstInitList([2])]))}
        \item 类型为 \inlinerust{IntArray([2][1])}
    \end{itemize}
    \item 函数\textbf{声明} \inlinec{int f(int a[][2]);} 定义的符号 \inlinec{main}
    \begin{itemize}
        \item 初始化器为 \inlinerust{None}
        \item 类型为 \inlinerust{Function(Int, [IntPointer([2])])}
    \end{itemize}
    定义的符号 \inlinec{a}
    \begin{itemize}
        \item 初始化器为 \inlinerust{None}
        \item 类型为 \inlinerust{IntPointer([2])}
    \end{itemize}
    \item 函数\textbf{定义} \inlinec{int f(int a) { return 0; } } 定义的符号 \inlinec{main}
    \begin{itemize}
        \item 初始化器为 \inlinerust{Function(["a"], Block([Return(0)]))}
        \item 类型为 \inlinerust{Function(Int, [Int])}
    \end{itemize}
\end{itemize}

实际实现中,初始化器和类型并不存储在 \inlinerust{Definition} 的内部,而是存储在一个哈希表中;\inlinerust{Definition} 仅存储指向哈希表的 \inlinerust{Handler}。

\inlinerust{Block}(复合语句)是语法树的基础。\inlinerust{Block} 是由一系列元素组成的列表,这些列表可以是:
\begin{itemize}
    \item \inlinerust{Definition}(局部变量定义);
    \item \inlinerust{Statement}(语句,即表达式(\inlinerust{Expr})、\inlinec{if}、\inlinec{while}、\inlinec{continue} 和 \inlinec{break});
    \item \inlinerust{Block}(复合语句),\inlinerust{Block} 套 \inlinerust{Block} 是很正常的。
\end{itemize}

\inlinerust{Expr}(表达式)是语法树的核心,仅摘录一部分:
\begin{minted}[linenos,frame=single]{rust}
pub enum Expr {
    Mul(Box<Expr>, Box<Expr>),
    Div(Box<Expr>, Box<Expr>),
    // ......
    Num(i32),
    Var(String),
    Func(String, Vec<Expr>),
    Array(String, Vec<Expr>, bool),
}
\end{minted}
\subsubsection{内存形式的汇编}
为了方便后端的优化和后处理,汇编以内存形式生成。部分代码摘录如下:
\begin{minted}[linenos,frame=single]{rust}
pub type RiscV = Vec<RiscVItem>;
pub enum RiscVItem {
    Label(String),
    Inst(Inst),
    Directive(Directive),
}
pub enum Inst {
    Blt(Reg, Reg, String),
    Bgt(Reg, Reg, String),
    // ......
    Slt(Reg, Reg, Reg),
    Snez(Reg, Reg),
}
\end{minted}
\subsection{主要设计考虑及算法选择}
\subsubsection{符号表}
见 \ref{semantic-checking}。
\subsubsection{寄存器分配}\label{reg-alloc}
\begin{itemize}
    \item 简单起见,全部变量都在栈上;
    \item 二元表达式计算时,变量被加载到 \texttt{t0}、\texttt{t1} 寄存器中,结果存储在 \texttt{t2} 寄存器,并且立即写入内存;
    \item 引用栈上变量,且偏移量大于 2048 字节时,偏移量被 \texttt{li} 到 \texttt{t3} 寄存器中,然后以 \texttt{sp} 或 \texttt{fp}(\texttt{s0}) 加之;
    \item 间接跳转时,跳转地址被 \texttt{la} 到 \texttt{t4},然后 \texttt{jr t4};
    \item 引用全局变量时,地址被 \texttt{la} 到 \texttt{t0};
    \item 函数调用时前 8 个函数参数加载到 \texttt{a0} \~{} \texttt{a7} 寄存器,被调用的函数立即将参数写入内存。
\end{itemize}
\subsubsection{优化}
在前端,对表达式进行代数化简,如 \inlinec{w = (x - x) * (y + z);} 会直接化简为 \inlinec{w = 0;};当 \inlinec{if} 或 \inlinec{while} 的条件表达式可以确定时(如 \inlinec{while (x - x)}),则生成的 IR 直接使用 \texttt{jump} 指令。

生成汇编时,对于栈上数据的引用,当栈大小小于 2048 字节时,使用 \texttt{sp} 寄存器访问;栈大小在 2048 \~{} 4096 字节时,高地址的数据使用 \texttt{fp}(\texttt{s0})寄存器访问。

在生成汇编后进行窥孔优化。具体包含:
\begin{itemize}
    \item 控制流化简,
\begin{minted}[linenos,frame=single]{asm}
    bne rs_1, rs_2, label_1
    j label_2
label_1:
\end{minted}
    优化为
    \begin{minted}[linenos,frame=single]{asm}
    beq rs_1, rs_2, label_2
label_1:
    \end{minted}
    \texttt{blt}、\texttt{bgt} 等指令的优化类似。而单独的:
    \begin{minted}[linenos,frame=single]{asm}
    j label_1
label_1:
    \end{minted}
        优化为
    \begin{minted}[linenos,frame=single]{asm}
label_1:
    \end{minted}
    \item 去除多余的 \texttt{lw}。
\begin{minted}[linenos,frame=single]{asm}
sw rs_1, offset(rs_2)
lw rd, offset(rs_2)
\end{minted}
    优化为
    \begin{minted}[linenos,frame=single]{asm}
sw rs_1, offset(rs_2)
mv rd, rs_1
    \end{minted}
    而
    \begin{minted}[linenos,frame=single]{asm}
sw rs_1, offset(rs_2)
lw rs_1, offset(rs_2)
\end{minted}
    直接优化为
    \begin{minted}[linenos,frame=single]{asm}
sw rs_1, offset(rs_2)
    \end{minted}
\end{itemize}
\section{编译器实现}
\subsection{预处理器}
预处理器负责的三个阶段全部采用有限状态自动机实现,并组织为生成器的形式。
\subsubsection{换行符替换}\label{stage-1}
不同平台上的换行符各不相同,Windows 默认为 \texttt{\textbackslash r\textbackslash n},mac 默认为 Windows 默认为 \texttt{\textbackslash r},而 Linux 默认为 \texttt{\textbackslash n}。为了保证编译器在这些情况下有一致的行为,第一阶段进行换行符替换。具体而言,是将 \texttt{\textbackslash r\textbackslash n} 或单独的 \texttt{\textbackslash r} 替换为 \texttt{\textbackslash n}。例如:
\begin{itemize}
    \item \texttt{\textbackslash r\textbackslash n} 应当替换为 \texttt{\textbackslash n}
    \item \texttt{\textbackslash r\textbackslash r\textbackslash n\textbackslash r} 应当替换为 \texttt{\textbackslash n\textbackslash n\textbackslash n}
    \item \texttt{\textbackslash r\textbackslash r\textbackslash r\textbackslash r} 应当替换为 \texttt{\textbackslash n\textbackslash n\textbackslash n\textbackslash n}
\end{itemize}
\subsubsection{行拼接}\label{stage-2}
根据 C 语言标准,凡在反斜杠出现于行尾(紧跟换行符)时,删除反斜杠和换行符,把两个物理源码行组合成一个逻辑源码行。\textbf{这属于我对 SysY 的扩展,而不是标准 SysY。}例如:
\begin{minted}[linenos,frame=single]{c}
int main() {
    int a = 1; // \
    int b;
    int b;     // \\\\
    int b;
    return a;
}
\end{minted}
被处理为:
\begin{minted}[linenos,frame=single]{c}
int main() {
    int a = 1; //     int b;
    int b;     // \\\    int b;
    return a;
}
\end{minted}
\subsubsection{去除注释}
预处理器按照 C 语言标准去除注释。
\subsection{解析器}
\subsubsection{第一次解析}
第一次解析中,使用 pest 将代码处理为语法树。因为 pest 设计上的局限性,此处语法树使用的是 pest 中 \inlinerust{Pairs<Rule>} 的形式。此外,这里还没有对运算符优先级做处理,表达式仍然表现为 \inlinerust{Pair} 流的形式。
\subsubsection{第二次解析}
遍历上述 \inlinerust{Pairs<Rule>} 的语法树,转换为 \inlinerust{TranslationUnit}。同时,在这一步使用 pest 提供的 \inlinerust{PrattParser<Rule>} 处理运算符优先级。
\subsubsection{语义检查}\label{semantic-checking}
遍历 \inlinerust{TranslationUnit},进行语义检查和常量表达式计算。

为支持作用域嵌套,ATC-X 的符号表设计为一个栈(\inlinerust{Vec<HashMap<String, Definition>>}),在查找符号时,优先查找作用域最靠内的符号。

因为 IR 中要求每个标识符是局部或全局唯一的,解析器在进行语义检查时,会同时进行名字重整,即为标识符加前缀 \texttt{\_I}(整型)、\texttt{\_P}(指针)或 \texttt{\_A}(数组)。
\begin{quotation}
    \textit{这个设计是不得已而为之。在设计的最初阶段,语法树里是没有符号表的。等到后面已经积重难返了。}
\end{quotation}

代码:
\begin{minted}[linenos,frame=single]{c}
int f(int a[]) {
    const int x[1] = {1};
    int y = 1, z[1] = {1};
    { int x = x[0], y = x; }
}
int g(int a[]) {
    const int x[1] = {1};
    int y = 1, z[1] = {1};
    { int x = x[0], y = x; }
}
\end{minted}
被重整化为:
\begin{minted}[linenos,frame=single]{c}
int f(int _Pa[]) {
    const int _Ax[1] = {1};
    int _Iy = 1, _Az[1] = {1};
    { int _Ix = _Ax[0], _I_Iy = _Ix; }
}
int g(int _Pa[]) {
    const int _A_Ax[1] = {1};
    int _Iy = 1, _Az[1] = {1};
    { int _Ix = _A_Ax[0], _I_Iy = _Ix; }
}
\end{minted}

考虑到 C 标准中将“下划线+大写字母”开头的标识符视为保留名,这样做是完全合理的。此外,可以看到两个函数中的局部常量数组做了特殊处理,因为生成 IR 时局部常量数组会被移动到全局。

实际实现中,第二次解析和语义检查是合并在一起的。除此之外,还会进行表达式化简,如 \inlinec{w = (x - x) * (y + z);} 会直接化简为 \inlinec{w = 0;}。
\subsection{IR 生成器}
\subsubsection{生成粗 IR}
前端直接生成文本形式的、非 SSA 的 Koopa IR。具体过程类似课程中的使用 SDT 生成中间表示。

在生成表达式的 IR 时,SysY 中的表达式被进一步分为三类:
\begin{enumerate}
    \item 左值表达式(lvalue),这种表达式计算的值是一个内存地址。例如,语句 \inlinec{x = 1;} 中的表达式 \inlinec{x} 是左值表达式;鉴于 ATC-X 对 SysY 做了语法扩展,\inlinec{(x = y) = z;} 中的\inlinec{x} 和 \inlinec{x = y} 都是左值表达式。
    \item 右值表达式(rvalue),这种表达式计算的值只是一个值。例如,\inlinec{x = (y += z);} 中的 \inlinec{z} 和 \inlinec{y += z} 都是右值表达式。
    \item 弃值表达式(dvalue),这种表达式计算的值会被舍弃。例如,语句 \inlinec{x = 1;} 中的 \inlinec{x = 1} 是弃值表达式,这个语句只会执行赋值的副作用而不产生值。
\end{enumerate}
\subsubsection{IR 后处理}
上述生成的 IR 称为粗 IR,因为它有很多不符合 Koopa IR 规范的情况。例如,代码:
\begin{minted}[linenos,frame=single]{c}
void f() { }
int main() {
    {
        while(1)
            continue;
        return 0;
    }
    return 0;
}
\end{minted}
会生成 IR:
\begin{minted}[linenos,frame=single]{text}
fun @f() {
%1:
}
fun @main(): i32 {
%3:
    jump %6
%6:
    jump %6
    jump %6
%7:
    ret 0
    ret 0
}
\end{minted}
可以看到,某些基本块存在连续的 \texttt{jump},某些基本块是空的。后处理就是要处理这些不合法的 IR,
分为两步:
\begin{enumerate}
    \item 在基本块的内部,第一个 \texttt{jump}、\texttt{br} 或 \texttt{ret} 后的所有指令被忽略;
    \item 如果某个基本块的结尾不是 \texttt{jump}、\texttt{br} 或 \texttt{ret},则根据函数返回值类型添加 \texttt{ret 0} 或 \texttt{ret}。
\end{enumerate}
上述 IR 后处理的结果为:
\begin{minted}[linenos,frame=single]{text}
fun @f() {
%1:
    ret
}
fun @main(): i32 {
%3:
    jump %6
%6:
    jump %6
%7:
    ret 0
}
\end{minted}
\subsection{后端}
后端使用 Koopa IR 库提供的 \inlinerust{koopa::front::Driver} 将文本形式的 IR 重建为内存形式;然后遍历内存形式的 IR,生成汇编。
\subsubsection{汇编生成}
寄存器分配的策略已在 \ref{reg-alloc} 阐述。

栈帧包含以下四部分:
\begin{enumerate}
    \item 为变量分配的空间;
    \item (如果这个函数不是叶函数),调用其他函数,且参数个数大于 8 时栈传参的空间;
    \item (如果这个函数不是叶函数),保存 \texttt{ra} 的空间;
    \item (如果这个函数不是叶函数,且栈帧大小大于 2048),保存 \texttt{fp}(\texttt{s0}) 的空间;
\end{enumerate}
\subsubsection{后处理与优化}
上述生成的汇编称为粗汇编,因为所有的分支指令都采用直接跳转;但分支指令的偏移量至多为 4096 字节,\texttt{j} 指令的偏移量至多为 1048576 字节,长距离跳转时很容易超出这一偏移量。后处理就是要将这些超出范围的直接跳转改为间接跳转。

作为一个例子,不妨假设下列 \texttt{ble} 中的 \texttt{label} 超出直接跳转的 4096 字节范围:
\begin{minted}[linenos,frame=single]{asm}
ble rs_1, rs_2, label
\end{minted}
如果 \texttt{label} 还在 1048576 字节范围内,则改为:
\begin{minted}[linenos,frame=single]{asm}
bgt rs_1, rs_2, label_2
j label
label_2:
\end{minted}
如果 \texttt{label} 已经超出 1048576 字节范围,则改为:
\begin{minted}[linenos,frame=single]{asm}
bgt rs_1, rs_2, label_2
la t4, label
jr t4
label_2:
\end{minted}
\subsection{工具库介绍}
\subsubsection[pest]{pest\footnote{\url{https://github.com/pest-parser/pest}}}
使用解析表达文法(PEG)的 parser。相比编译原理实践文档中推荐的 lalrpop\footnote{\url{https://github.com/lalrpop/lalrpop}} 或 flex/bison,pest 的编写和调试都更加简单。
\subsubsection[koopa]{koopa\footnote{\url{https://github.com/pku-minic/koopa}}}
编译原理课程使用的 IR。本项目中只使用 Koopa IR 的非 SSA 版本。
\subsubsection[generator]{generator\footnote{\url{https://github.com/Xudong-Huang/generator-rs}}}
Rust 的第三方生成器/协程库。Rust 语言的 coroutine 还在 unstable 阶段,只好用第三方库代替了。
\section{语言扩展汇总}\label{lang-ext}
除标准 SysY 语言外,ATC-X 扩展了如下功能:
\begin{enumerate}
    \item 二进制整数字面量,如:
    \begin{minted}[linenos,frame=single]{c}
int a = 0b11011111101010010; // a = 114514
    \end{minted}
    \item 自定义中缀运算符,例如:
    \begin{minted}[linenos,frame=single]{c}
int foo(int x) { /* do something */ }
int foo2(int x) { /* do something */ }
int bar(int x, int y) { /* do something */ }
int main() {
    int x = 2, y = 2;
    x ~bar~ y;            // 等价于 bar(x, y);
}
    \end{minted}
    \item 管道运算符与成员访问运算符,例如:
    \begin{minted}[linenos,frame=single]{c}
int foo(int x) { /* do something */ }
int foo2(int x) { /* do something */ }
int bar(int x, int y) { /* do something */ }
int main() {
    int x = 2, y = 2;
    x |> foo |> foo2;     // 等价于 foo2(foo(x));
    x |> foo() |> bar(y); // 等价于 bar(y, foo(x));
    bar(x) <| y;          // 等价于 bar(x, y);
    ((x.bar(y)) ~bar~ 114)
      |> foo;             // 等价于 foo(bar(bar(x, y), 114));
}
    \end{minted}
    \item 函数声明语法,允许在函数签名中不指明参数名称,对于没有参数的函数也允许使用 \inlinec{void} 为参数列表。例如:
    \begin{minted}[linenos,frame=single]{c}
int foo(int);
int bar(int, int);
int main(void) { return foo(bar(114, 514)); }
int bar(int x, int) { return x + 1919; }
    \end{minted}
    \item 各种位运算符,如 \inlinec{<<}、\inlinec{^} 和 \inlinec{~}。
    \item 赋值被视为又结合的中缀运算符而不是语句,且赋值运算符返回左值。
    \item 各种复合赋值运算符,如 \inlinec{+=},\inlinec{*=}。
    \item 自增自减运算符,其中前缀自增自减返回左值。例如 \inlinec{++x = 1;}
    \item 以常量下标取得的常量数组元素可以参与编译期常量表达式求值,如:
    \begin{minted}[linenos,frame=single]{c}
int main() {
    const int a[1] = {10};
    const int x = a[0]; // x == 10
    int y[a[0]];        // y 的类型是 int[10]
    return 0;
}
    \end{minted}
\end{enumerate}
\section{计划中但未实现的功能}
\begin{enumerate}
    \item 编译期函数;
    \item 允许非常量表达式初始化全局变量;
    \item 寄存器分配。
\end{enumerate}
\section{实习总结}
\subsection{收获和体会}
使用 Rust 实现编译 Lab 极大地提升了我的 Rust 水平(虽然还是天天骂 Rust 编译器);此外,使用 PEG 而非 CFG,也扩展了我的知识面。
\subsection{对编译 Lab 的建议}
\begin{enumerate}
    \item 修复 Koopa IR 中基本块的名称不能为 “\texttt{\%} + 数字”的 bug;
    \item 编译 Lab 支持用更多语言编写,如 Scala、F\# 等函数式语言。(真的不想再碰 Rust 了……)
\end{enumerate}
\section{其他}
本文的标题致敬 Wolfram Mathematica 参考文档《关于内部实现的一些注释》。
